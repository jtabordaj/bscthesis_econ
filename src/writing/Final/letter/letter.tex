
\documentclass[11pt]{letter} 
\usepackage{newcent} 
\usepackage[spanish]{babel}
\topmargin=-0.5in 
\textheight=9.5in 
\oddsidemargin=-10pt 
\textwidth=6.5in
\let\raggedleft\raggedright 
\begin{document}

\begin{letter}{ Prof. Dr. Jose Luis Ramos Ruiz \\
Departamento de Economía \\
Universidad del Norte \\
Barranquilla, Colombia} 

\begin{flushleft}
\large Juan Jose Taborda Nuñez \\ 
\vspace{20pt} \hrule height 1pt 
jtabordaj@uninorte.edu.co \\ Barranquilla, Colombia\\ (300) 710-9772
\end{flushleft} 
\vspace{30pt}

\signature{ Juan Jose Taborda Nuñez \\ Estudiante de Economía} 


\opening{Estimado evaluador,} 

Le agradezco el tiempo otorgado para evaluar mi tesis de grado \textit{A characterization of Colombian industries under Schumpeter's patterns of innovation}, elaborada para optar por el titulo de Economista. A continuación, he desglosado mis decisiones sobre las modificaciones sugeridas:

\begin{enumerate}
	\item "\textit{Es importante que se incorpore en el marco teórico autores utilizados en el procedimiento metodológico, debido a que son autores que presentan argumentos sobre el tema de la innovación.}". \textbf{Se acepta}.
	\item "\textit{Si bien realiza la clasificación de las empresas en CG1 y CG2 con los preceptos Schumpeterianos, no es suficiente para entender los resultados de la clusterización. Por tanto, es deseable incorporar otros autores que complementen la fundamentación teórica de Schumpeter. El estudiante menciona a Malerva, quien tiene un libro sobre sistemas de innovación regional, el cual puede ayudar a resolver conceptualmente (...)}".  \textbf{Se acepta}, se ha introducido una sección de Sistemas de Innovación.
	\item "\textit{En la misma vía de encontrar 2 clúster, es importante relacionar las empresas en el contexto regional. No es útil una clusterización nacional, cuando la localización geográfica de la industria puede estar relacionada con características física del territorio, incluso de dotación de factores.}". \textbf{Se acepta}, se han introducido conceptos espaciales y se han analizado sus interacciones  con los patrones de Schumpeter.
	\item "\textit{Los resultados son débiles en la medida que no incorpora explicaciones a la luz de la teoría de Schumpeter y de la revisión de la literatura. Por tanto, se sugiere integrar los resultados de la clusterización con las explicaciones teóricas de Schumpeter y de otros autores tales como Pavitt, Nelson, Orsenigo y Malerba, entre otros.}". \textbf{Se acepta}.
	\item "\textit{En la perspectiva de promover la apropiación social del conocimiento sugiero, en lo posible, hacer una clusterización de esas industrias a nivel regional, dado que las recomendaciones de política podrían incorporarse en los planes territoriales de desarrollo y en los planes estratégico-departamentales de CTeI (PEDCTI)"}. \textbf{Se acepta}, se ha diseñado una sección entera que incorpora mapas discriminados por departamento.
	\item "\textit{Según la literatura económica, la escogencia del método de clusterización depende de las unidades de medida de las variables. El trabajo propuesto utiliza la Distancia Euclídea, la cual presenta algunas restricciones a saber (...) Es conveniente explorar las alternativas de clusterización del trabajo realizado para la industria manufacturera bajo la medición de la DISTANCIA DE MAHALANOBIS}". \textbf{No se acepta}, se ha optado por mantener el método \textit{k-means} pero estandarizando los datos, y se ha elaborado una sección donde se defiende matemáticamente el uso de datos estandarizados.
	\item "\textit{El informe esta redactado en primera persona. Un informe de investigación debe redactase en tercera persona}". \textbf{Se acepta}.
\end{enumerate}

Le agradezco una vez mas por sus comentarios, han sido cruciales para mejorar la calidad académica del trabajo. Sin más que agregar, me suscribo atentamente.

\closing{Cordialmente,}
\end{letter}

\end{document}